\documentclass[compsoc]{IEEEtran}
\IEEEoverridecommandlockouts
\usepackage{cite}
\usepackage{caption}
\usepackage{amsmath,amssymb,amsfonts}
\usepackage{algorithm}
\usepackage[noend]{algpseudocode}
\usepackage{textcomp}
\usepackage{xcolor}
\usepackage{comment}
\usepackage{array}
\def\BibTeX{{\rm B\kern-.05em{\sc i\kern-.025em b}\kern-.08em
    T\kern-.1667em\lower.7ex\hbox{E}\kern-.125emX}}
\usepackage{stfloats}
\usepackage{float}
\usepackage{hhline}
%\usepackage{caption}
\usepackage{booktabs}
\usepackage[export]{adjustbox}

\usepackage[caption=false,font=footnotesize]{subfig}
\usepackage{graphicx,lettrine}

\graphicspath{ {images/} }

\usepackage{pgfplots}
\pgfplotsset{width=7cm,compat=1.8}
\captionsetup{format=default, font=footnotesize, labelfont=bf}


%%%%%%%%%%%%%%%%
\begin{document}
%%%%%%%%%%%%%%%%
\title{Interactive Cryptocurrency Prediction using LSTM Recurrent Neural Networks }

\author{Vishal~Patil and
        James~Rivett
\thanks{V. Patil and J. Rivett are with the College of Computing and Software Engineering, Kennesaw State University, Marietta, GA, 30060 USA. Emails: vpatil@students.kennesaw.edu and jrivett@students.kennesaw.edu}}

%%%%%%%%%%%%%%%
\IEEEtitleabstractindextext{%
    
    \begin{abstract}
    \emph{Cryptocurrency} is an exciting new financial resource that results from financial innovation in computer science. Cryptocurrency is utilized to make exchanges anonymously and safely over the web. Despite this original use case, various digital assets have become subjects of investment just as often, if not more, than a subject of the transaction. That said, cryptocurrency price forecasting is difficult due to volatility and insufficient analytical support. This project proposes using a Recurrent Neural Network (RNN) with Long Short-Term Memory (LSTM) to predict the future prices of the top ten cryptocurrencies and compares the accuracy of a mixture of different models and hyper-parameters to yield the most accurate and reliable results. Comparing the various models and parameters will find out which ones best fit the test data over a given period and offer suitable predictions for the future. This application will provide insight and be an excellent stepping-stone for new retail investors looking to make their first big break in cryptocurrency. By implementing an easy-to-use interface, Investors can see visualized comparisons between cryptocurrencies. Moreover, the proposed framework can also be used in other applications where high volatility and scarcity of data are the main characteristics.
    \end{abstract}
    %%%%%%%%%%%%%%%%
    \begin{IEEEkeywords}
    Crypto, Bitcoin, LSTM, RNN, Neural Network, Prediction, Investing, GUI
    \end{IEEEkeywords}
    %%%%%%%%%%%%%%%%%%%%%%%
    }
    
\maketitle
\IEEEdisplaynontitleabstractindextext
\section{Introduction}
\lettrine{A}{ccording} to Forbes Magazine\cite{tretina_2022}, the aggregate Market Capitalization of the top 10 Cryptocurrencies is over 2 trillion USD as of February 2022. Entrepreneurs, large corporations, and retail investors from all over the income spectrum are looking to capitalize on this new, confusing, and promising technology. Much Like stocks, Cryptocurrencies have inherent qualities related to their trade that allows investors to decide whether to buy or sell. Features like the market cap, volume, and even social sentiment influence the price of any given digital commodity. While it is easy to identify the critical aspects, the values of those aspects are what tell the trained venture capitalist when to invest. That said, the volatility of these assets often makes it hard to pin down precisely what those values are. 

There has been a tremendous amount of research done on the topic of predicting stocks and crypto prices \cite{jaquart_dann_weinhardt_2021} \cite{edsdoj.7cab738d48645d094937050d942431420211001} \cite{edseee.959642920211025}, mostly due to easy and free access of large scale, well-annotated data-sets. With the recent rising popularity of blockchain technology, cryptocurrencies have attracted considerable attention worldwide. Anyone with a few dollars in their pocket could make a considerable profit on a Crypto investment if they had an idea of how it would move in the future.

The market is diverse, and there are over twelve thousand cryptocurrencies to choose from. Major cryptocurrencies can be bought using fiat currency in several online exchanges. Crypto coins are of different types, e.g., Mining-based(new coins generated by solving problems), Stablecoins(linked to fiat currency), Security tokens(resemble stocks and promise equity of company), Meme coins(inspired by jokes), and Utility Tokens(used to provide services in a network)\cite{altcoin}. This report features only the top ten on Coinbase.com's home page. Bitcoin was designed as a medium of exchange, while Ethereum is a public, blockchain-based distributed computing platform featuring smart contract (scripting) functionality. Tether (USDT) is a blockchain-based stable cryptocurrency backed by an equivalent amount of U.S. dollars.\cite{frankenfield_2022} Dogecoin (DOGE) is a peer-to-peer, open-source cryptocurrency and is considered a sarcastic meme coin made famous by Twitter and Reddit.  

Price prediction can be accomplished through Long Short-Term Memory(LSTM) Recurrent Neural Networks. LSTMs are widely used for sequence prediction problems [9] and have proven highly effective in such situations. They work so well because LSTM layers store past important information back to a defined point, and forget the information behind it. In short, an LSTM cell can have a 'memory' that is exactly as long as the user defines (implemented by a global LOOK\_BACK variable). This feature explains why this architecture can successfully capture patterns in series-dependent data such as long texts, audio recordings, and obviously asset prices.

The primary purpose of this ML program is to construct a reliable prediction model that investors can rely on based on documented cryptocurrency prices. AI approaches will potentially reveal examples and insights we had not seen before, and these can be used to make unerringly exact expectations, using features like the most recent declarations of an organization, their quarterly income figures, etc.
We will work with published information regarding freely recorded cryptocurrency prices in this report. We will use a combination of AI calculations to forecast the selected cryptocurrency price with LSTM. This paper aims to answer: ‘How can machine learning algorithms help investors and decision-makers predict cryptocurrency prices?’ and ‘What model best predicts future prices?’
%%%%%%%%%%%%%%%%%%%%%%%

\section{Related works} 

Our baseline model came from Sudharsan Ravichandiran, a Data Scientist and Mathematician from India. He had a simple implementation of his model where he'd obtain data in a CSV file and run said sequential model with LSTM layers to test and train data.\cite{ravichandiran_2018}
\begin{figure}[h!]
\includegraphics[width=0.5\textwidth]{img/Sudharsan-graph}
\caption{Sudharsan's final graph with legend}
\end{figure}

His graphs were lackluster in data presentation and was missing several crucial aspects, such as proper axis labels and were subject to poor scaling. Ravichandiran's algorithm trained an old data set consisting only of bitcoin and had to rely on a static CSV file to import data. In-spite of these issues, his implementation did provide an excellent baseline from which we could start our study.

%%%%%%%%%%%i%%%%%%%%%%%%

\section{Approach (and technical correctness)}

We implemented our solution using Python programming language and various libraries that make the process relatively discrete. 
\begin{figure}[h!]
\includegraphics[width=0.5\textwidth]{img/libraries}
\caption{Libraries used for framework}
\end{figure}

The GUI library we used is called appJar, a relatively easy-to-start framework that allows a programmer to create UI and quickly access its inputs. Libraries such as NumPy, Pandas, MatPlotLib, and DateTime take care of housekeeping tasks such as array operations, data framing, plotting, and date manipulation, respectively. SciKitLearn, or 'sklearn,' is the main driver for the neural network. This library contains much of the necessary functionality to build and train a neural network. From the SciKitLearn framework, we import MinMaxScaler for data flattening and Mean Squared Error to serve as our primary performance metric. In addition, we use the sequential model along with dense and LSTM layers from Keras to serve as our functional Neural Network architecture.

Logically, the application begins by gathering the requested data. While there are various sources from which one could get standardized price data, an API best serves this use case due to its availability, modularity, and consistency. For this implementation, we elected to use CoinAPI. A parameterized URL endpoint for CoinAPI\cite{coinapi.io} is passed to Pandas' integrated CSV parser, which handles the HTTP request. These parameters are obtained through the User Interface, made in AppJar.
The interface prompts the user for the desired crypto, the date range to predict, and the trading interval to poll. Once that information is known, the program then makes an HTTP request to CoinAPI\cite{coinapi.io}, who will return a CSV file with the desired data defined by the user.
\begin{figure}[h!]
\includegraphics[width=0.5\textwidth]{img/Interface.jpg}
\caption{User Interface for cryptocurrency selection and DateTime input}
\end{figure}

Once the CSV file for the desired data has been properly obtained, the application selects the opening price of each period as the single-dimensional data frame. The MinMaxScaler from sklearn then flattens the data frame into values between 0 and 1. Next, the Training/Testing split is selected based on a global variable called 'TT\_SPLIT'. Now that the training and test sets are split, we define a function called 'create\_dataset()' that creates a rolling section of the data frame based on a global variable called 'LOOK\_BACK'.

\includegraphics[width=0.5\textwidth]{img/data-preprocessing-flow.jpg}

That rolling data is then passed to a sequential neural network model from Keras\cite{chollet_2020}, the layers of which are structured as follows: An LSTM Layer with an input shape of (1, LOOK\_BACK) with an output shape of 4, LOOK\_BACK, then a generic Dense layer with an input dimensionality of 4, and output of 1, which will be inverse transformed to return the price value.
\includegraphics[width=0.5\textwidth]{img/neural-network-architecture.jpg}

Long Short-Term Memory (LSTM) helps make training converge quickly and detect long-term reliant parameters in the data\cite{geron_2019}. It also gives us the flexibility to fine-tune the number of previous data points the network should consider. In addition, there are many other parameters such as lookback and batch size that can be changed via global variables in the code. This allows us to experiment with different combinations of parameters that can most accurately derive predictions for price data against the validation set.
%%%%%%%%%%%%%%%%%

\section{Experimental results (and technical correctness)} 
In our first test with the baseline model and parameters, the selected cryptocurrency is Bitcoin, the number of epochs is 100, with a time range from the 1st of January, 2020 to the 1st of December, 2020. In return, we get fitting lines which predict the testing data over trading intervals of 12 hours.
\includegraphics[width=0.5\textwidth]{img/test-with-sudharsan-params.png}

While the results are better than random guesses, they are not as accurate as they could be. Especially when compared to other implementations, an RMSE in the thousands is less than optimal, regardless of the scaling experienced due to the coin's high price. There could be many reasons behind this, one of which is that this date range contains a large hike in price that leads to under-fitting, in the validation set since that hike wasn't experienced in the training set. In addition, there are many other things about this implementation's inherent logic that can be changed to make it more accurate.
To begin implementing these improvements, we first select a date range with a wider variety of market events. The spike experienced in the previous data set isn't the inherent problem itself, but rather that it wasn't seen in the training phase. By expanding the end of our data selection to a more current date (2022-02-01) the model will train itself to react to more drastic market events such as this. 
\includegraphics[width=0.5\textwidth]{img/expanded-date-range.png}

While the RMSE did increase on this specific test, it is evident that the model can better account for significant market occurrences if trained to do so. Another parameter to change in this situation is the number of epochs used for training. While 100 epochs would be sufficient for an extensive data set as the one used in Ravichandiran's implementation, we could achieve better accuracy with more epochs. This change also poses less risk for over-fitting with our data set since the range of values is smaller. We then ran many tests to determine the best amount of epochs to provide the best performance. The number of epochs was increased in increments of 100 up to 600 in each test, while the neural network was tested twice at each increment. The application then displays a plot of the actual data overlayed with the predictions of both the training and validation sets. The RMSE for each set is also displayed at the bottom as the performance metric. The averages of those RMSE scores between the two tests were then recorded and passed to separate graphs for comparison. 
\includegraphics[width=0.5\textwidth]{img/exp/EPOCH-200-run1.png}
\includegraphics[width=0.5\textwidth]{img/exp/EPOCH-300-run1.png}
\includegraphics[width=0.5\textwidth]{img/exp/EPOCH-400-run1.png}
\includegraphics[width=0.5\textwidth]{img/exp/EPOCH-400-run2.png}
\includegraphics[width=0.5\textwidth]{img/exp/EPOCH-500-run2.png}
\includegraphics[width=0.5\textwidth]{img/exp/EPOCH-600-run2.png}
While the graphs look similar because they are derived from the same data set, one can discern minor differences between how the prediction lines fit the actual data between amounts of epochs. One may expect that accuracy linearly increases with epochs, but this is not always the case. Usually, each application has an optimal amount that makes the best model for making predictions against the validation set. This data set, in particular, is a good demonstration of this phenomenon since there are a variety of prices and market events.
\includegraphics[width=0.5\textwidth]{img/exp/accuracy-epoch.png}

The next parameter we sought to optimize was the 'LOOK\_BACK' variable that determines the size of the rolling data frame the LSTM layer considers. While one could assume that a higher look back would account for more varied market occurrences, but that quickly leads to over-fitting in excess application. One might also correctly assume that a shorter look back would result in lower error rates. While that is the case, it defeats the inherent purpose of the neural network, as it will no longer be able to predict as far in the future as it could before. 

Before running official experiments, we determined a look back larger than 10 would lead to over-fitting due to the volatility of the data set, so we worked downward from there. We also determined that a look back of less than 6 data points would be relatively unhelpful to prospective investors. That said, we did run an experiment with a look back range of 2 data points to demonstrate the drastic increase in accuracy that is experienced with smaller look back values. 
\includegraphics[width=0.5\textwidth]{img/exp/LOOKBACK-10.png}
\includegraphics[width=0.5\textwidth]{img/exp/LOOKBACK-9.png}
\includegraphics[width=0.5\textwidth]{img/exp/LOOKBACK-8.png}
\includegraphics[width=0.5\textwidth]{img/exp/LOOKBACK-7.png}
\includegraphics[width=0.5\textwidth]{img/exp/LOOKBACK-6.png}
\includegraphics[width=0.5\textwidth]{img/exp/LOOKBACK-2.png}
Although precision did increase as look back decreased as aforementioned, that's not always a useful thing to do. Since it is a good balance between future-proofing and accuracy, we selected to stay with a look back frame of 6 data points.
\includegraphics[width=0.5\textwidth]{img/exp/accuracy-look-back.png}
The graphic above makes it evident that a user will experience diminishing returns with an excessively large look back, but will gain accuracy when decreasing the size.

The last parameter we sought to optimize was the model's batch size. This parameter determines how many replications of the rolling data set the network will use in model training. These batch sizes are defined in powers of 2, so we began at 16 and went up to Ravichandran's value in the original model which is 256.
\includegraphics[width=0.5\textwidth]{img/exp/EPOCH-200-run1.png}
\includegraphics[width=0.5\textwidth]{img/exp/EPOCH-300-run1.png}
\includegraphics[width=0.5\textwidth]{img/exp/EPOCH-400-run1.png}
\includegraphics[width=0.5\textwidth]{img/exp/EPOCH-400-run2.png}
\includegraphics[width=0.5\textwidth]{img/exp/EPOCH-500-run2.png}
\includegraphics[width=0.5\textwidth]{img/exp/EPOCH-600-run2.png}
Again, this is an example of parameter convergence and divergence. While the score does improve at first, there is a point at which the score begins to rise again. From these results, we determined the optimal batch size value for this particular data set was 32.
\includegraphics[width=0.5\textwidth]{img/exp/accuracy-batch-size.png}

%\includegraphics{}
%%%%%%%%%%%%%%%%%%%%

\section{Conclusion} 
In this project, we constructed and used machine learning models to predict the prices of ten types of cryptocurrencies with a mixture of different parameterizations to yield accurate and reliable results. Comparing the various models, we found out what fits the actual test data the best and offers accurate predictions for the future. While this functionality alone is useful, the application truly expands its use cases with the implementation of an easy-to-use interface. This application provides great insight and is an excellent stepping-stone for us and others interested in cryptocurrency. With enough computation power, anyone is able to get visualized projections and comparisons and pursue further studies on this topic that were not previously as accessible due to the lack of simplicity in other implementations. As a plan for future work, we could investigate other factors that might affect the cryptocurrency market prices, such as social media sentiment, relative volatility, and various other flagship market indicators.

%%%%%%%%%%%%%%%%%%%%%%%
\bibliographystyle{IEEEtran}
\bibliography{References} % (15 - 25) references
\nocite{*}
\end{document}
